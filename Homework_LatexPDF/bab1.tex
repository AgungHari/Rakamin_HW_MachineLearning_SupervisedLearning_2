\section{Pendahuluan}
\subsection{Latar Belakang}

Perusahaan telekomunikasi di era digital saat ini menghadapi tantangan besar dalam mempertahankan pelanggan. Salah satu permasalahan utama yang sering dihadapi adalah fenomena \textit{customer churn}, yaitu kondisi ketika pelanggan memutuskan untuk berhenti berlangganan suatu layanan. Tingginya tingkat churn dapat berdampak signifikan terhadap pendapatan perusahaan, mengingat biaya untuk memperoleh pelanggan baru umumnya lebih tinggi dibandingkan mempertahankan pelanggan yang sudah ada.

Memahami faktor-faktor yang mempengaruhi churn serta mampu memprediksi pelanggan mana yang berpotensi churn menjadi sangat penting bagi perusahaan. Dengan prediksi yang akurat, perusahaan dapat melakukan intervensi yang tepat, seperti memberikan penawaran khusus atau meningkatkan kualitas layanan, guna mencegah kehilangan pelanggan.

Melalui pemanfaatan data historis pelanggan, seperti data demografi, riwayat penggunaan layanan, dan perilaku pembayaran, teknologi \textit{data science} dan \textit{machine learning} memungkinkan perusahaan untuk membangun model prediksi churn yang efektif. Model ini dapat menjadi alat bantu pengambilan keputusan strategis, sehingga perusahaan dapat fokus pada segmen pelanggan yang paling berisiko churn dan merancang strategi retensi yang lebih efisien.

Tugas ini berfokus pada pembangunan model klasifikasi untuk memprediksi churn pelanggan pada perusahaan telekomunikasi berdasarkan berbagai fitur pelanggan. Diharapkan hasil dari proyek ini dapat memberikan insight dan rekomendasi bisnis yang relevan untuk meningkatkan retensi pelanggan.

\subsection{Tujuan}
Tujuan dari tugas ini adalah diantara lain untuk:

\begin{itemize}
    \item Membangun model klasifikasi untuk memprediksi churn pelanggan pada perusahaan telekomunikasi.
    \item Menganalisis faktor-faktor yang berkontribusi terhadap churn pelanggan.
    \item Memberikan rekomendasi strategis untuk meningkatkan retensi pelanggan berdasarkan hasil analisis dan model yang dibangun.
\end{itemize}

\subsection{Batasan atau Ruang Lingkup}
Batasan masalah dalam tugas ini mencakup:
\begin{itemize}
    \item Menggunakan dataset yang diberikan oleh rakamin.
    \item Penggunaan model klasifikasi untuk memprediksi churn.
    \item Analisis dilakukan pada fitur-fitur yang tersedia dalam dataset, tanpa melakukan pengumpulan data tambahan.
\end{itemize}

\subsection{Manfaat}
Manfaat dari tugas ini adalah membuat model klasifikasi yang dapat digunakan untuk memprediksi churn pelanggan.
